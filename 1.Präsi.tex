
\documentclass{beamer}\usepackage[]{graphicx}\usepackage[]{color}
%% maxwidth is the original width if it is less than linewidth
%% otherwise use linewidth (to make sure the graphics do not exceed the margin)
\makeatletter
\def\maxwidth{ %
  \ifdim\Gin@nat@width>\linewidth
    \linewidth
  \else
    \Gin@nat@width
  \fi
}
\makeatother

\definecolor{fgcolor}{rgb}{0.345, 0.345, 0.345}
\newcommand{\hlnum}[1]{\textcolor[rgb]{0.686,0.059,0.569}{#1}}%
\newcommand{\hlstr}[1]{\textcolor[rgb]{0.192,0.494,0.8}{#1}}%
\newcommand{\hlcom}[1]{\textcolor[rgb]{0.678,0.584,0.686}{\textit{#1}}}%
\newcommand{\hlopt}[1]{\textcolor[rgb]{0,0,0}{#1}}%
\newcommand{\hlstd}[1]{\textcolor[rgb]{0.345,0.345,0.345}{#1}}%
\newcommand{\hlkwa}[1]{\textcolor[rgb]{0.161,0.373,0.58}{\textbf{#1}}}%
\newcommand{\hlkwb}[1]{\textcolor[rgb]{0.69,0.353,0.396}{#1}}%
\newcommand{\hlkwc}[1]{\textcolor[rgb]{0.333,0.667,0.333}{#1}}%
\newcommand{\hlkwd}[1]{\textcolor[rgb]{0.737,0.353,0.396}{\textbf{#1}}}%

\usepackage{framed}
\makeatletter
\newenvironment{kframe}{%
 \def\at@end@of@kframe{}%
 \ifinner\ifhmode%
  \def\at@end@of@kframe{\end{minipage}}%
  \begin{minipage}{\columnwidth}%
 \fi\fi%
 \def\FrameCommand##1{\hskip\@totalleftmargin \hskip-\fboxsep
 \colorbox{shadecolor}{##1}\hskip-\fboxsep
     % There is no \\@totalrightmargin, so:
     \hskip-\linewidth \hskip-\@totalleftmargin \hskip\columnwidth}%
 \MakeFramed {\advance\hsize-\width
   \@totalleftmargin\z@ \linewidth\hsize
   \@setminipage}}%
 {\par\unskip\endMakeFramed%
 \at@end@of@kframe}
\makeatother

\definecolor{shadecolor}{rgb}{.97, .97, .97}
\definecolor{messagecolor}{rgb}{0, 0, 0}
\definecolor{warningcolor}{rgb}{1, 0, 1}
\definecolor{errorcolor}{rgb}{1, 0, 0}
\newenvironment{knitrout}{}{} % an empty environment to be redefined in TeX

\usepackage{alltt}
\usetheme{Singapore}
\usepackage{german}
\usepackage[utf8]{inputenc}
\IfFileExists{upquote.sty}{\usepackage{upquote}}{}
\begin{document}

\title{Kleinräumige extrapolation von Umfragedaten}   
\author{Kai Husmann, Alexander Lange} 
\date{\today}
\logo{\includegraphics[scale=0.25]{logo_goe}}

\begin{frame}
\titlepage
\end{frame}

\begin{frame}
\frametitle{Inhaltsverzeichnis}\tableofcontents
\end{frame}


\section{Abschnitt Nr.1} 
\begin{frame}\frametitle{Titel} 
Die einzelnen Frames sollte einen Titel haben 
\end{frame}

\subsection{Unterabschnitt Nr.1.1  }
\begin{frame}\frametitle{Testtitel}
Denn ohne Titel fehlt ihnen was
\end{frame}


\section{Abschnitt Nr.2} 
\subsection{Listen I}
\begin{frame}\frametitle{Aufz\"ahlung}
\begin{itemize}
\item Einf\"uhrungskurs in \LaTeX  
\item Kurs 2  
\item Seminararbeiten und Pr\"asentationen mit \LaTeX 
\item Die Beamerclass 
\end{itemize} 
\end{frame}

\begin{frame}\frametitle{Aufz\"ahlung mit Pausen}
\begin{itemize}
\item  Einf\"uhrungskurs in \LaTeX \pause 
\item  Kurs 2 \pause 
\item  Seminararbeiten und Pr\"asentationen mit \LaTeX \pause 
\item  Die Beamerclass
\end{itemize} 
\end{frame}

\subsection{Listen II}
\begin{frame}\frametitle{Numerierte Liste}
\begin{enumerate}
\item  Einf\"uhrungskurs in \LaTeX 
\item  Kurs 2
\item  Seminararbeiten und Pr\"asentationen mit \LaTeX 
\item  Die Beamerclass
\end{enumerate}
\end{frame}
\begin{frame}\frametitle{Numerierte Liste mit Pausen}
\begin{enumerate}
\item  Einf\"uhrungskurs in \LaTeX \pause 
\item  Kurs 2 \pause 
\item  Seminararbeiten und Pr\"asentationen mit \LaTeX \pause 
\item  Die Beamerclass
\end{enumerate}
\end{frame}

\section{Abschnitt Nr.3} 
\subsection{Tabellen}
\begin{frame}
\frametitle{Tabellen}
\begin{tabular}{|c|c|c|}
\hline
\textbf{Zeitpunkt} & \textbf{Kursleiter} & \textbf{Titel} \\
\hline
WS 04/05 & Sascha Frank &  Erste Schritte mit \LaTeX  \\
\hline
SS 05 & Sascha Frank & \LaTeX \ Kursreihe \\
\hline
\end{tabular}
\end{frame}


\begin{frame}
\frametitle{Tabellen mit Pause}
\begin{tabular}{c c c}
A & B & C \\ 
\pause 
1 & 2 & 3 \\  
\pause 
A & B & C \\ 
\end{tabular} 
\end{frame}


\section{Abschnitt Nr.4}
\subsection{Bl\"ocke}
\begin{frame}\frametitle{Bl\"ocke}

\begin{block}{Blocktitel}
Blocktext 
\end{block}

\begin{exampleblock}{Blocktitel}
Blocktext 
\end{exampleblock}


\begin{alertblock}{Blocktitel}
Blocktext 
\end{alertblock}
\end{frame}

\section[Quellen]{Referezen}
\begin{frame}\frametitle{Quellen \& Literatur}

\begin{thebibliography}{9}
\bibitem[Beamerpaket]{paket} \emph{Beamer Paket} \\ 
\text{http://latex-beamer.sourceforge.net/}
\bibitem[Beamerdokumentation]{doku} \emph{User's Guide to the Beamer} 
\bibitem[Dante]{dante} \emph{DANTE e.V.} \text{http://www.dante.de}   
\end{thebibliography}


\end{frame}



\end{document}
